% !TEX root =  paper.tex

\section{Empirical Evaluation}\label{sec:evaluation}

We consider the following research questions:

\noindent
\textbf{RQ\textsubscript{1}:} How prevalent are the various breakage scenarios in practice?

\noindent
\textbf{RQ\textsubscript{2}:} How do visual-based and DOM-based test repair approaches compare in terms of effectiveness?

\noindent%\quad\hangindent=0.35cm
\textbf{RQ\textsubscript{3}:} this is research question 3

\noindent
In our study, the visual-based approach is represented by \tool, and the DOM-based approach is represented by \water.

\subsection{Subjects}\label{sec:subjects}

We selected four open source web applications for which Selenium test cases were available. Such applications span different application domains and have been used in the context of previous research on web testing~\cite{}.

\subsection{Procedure}\label{sec:procedure}

\subsubsection{Metrics}

\begin{enumerate}
\item The number of \textit{true positives} (TP):The number of samples that have been correctly identified as being the class being sought (e.g. that a face exists at a particular location).
\item The number of \textit{false positives} (FP): The number of samples that have been incorrectly identified as being the class being sought (e.g. that a face exists at a particular location when there is no face present).
\item The number of \textit{true negatives} (TN): The number of samples that have been correctly identified as not being the class being sought (e.g. that a face does not exist at a particular location).
\item The number of \textit{false negatives} (FN): The number of samples that have been incorrectly identified as not being the class being sought (e.g. that a face does not exist at a particular location when there is a face present at that location).
\item Recall is the percentage of the objects being sought which have been successfully located. Recall = TP / (TP + FN)
\item Precision is the percentage of the positive classifications which are correct (i.e. are not false alarms). Precision = TP / (TP + FP)
\item Accuracy is the percentage of the total samples which are correct. Accuracy = TP + TN / TotalSamples
\end{enumerate}

\noindent
(1)~use \textit{many} (10-15) subjects (e.g., test suites) and only two (2) major releases. Advantage is that there is threat to internal/construct validity where applying a different repair is concerned. There is the TTV of the selection of the releases. \\

\noindent
(2)~use \textit{few} (4-5) subjects (e.g., test suites) but span multiple releases  (30-50) releases. Disadvantage is that the kind of repair applied during the evolution can greatly influence the breakage occurrences. \\

\noindent
(3)~user study to measure how effective the tool is in the identification of the root cause, and how much time it allows to save.

\subsection{Results}\label{sec:results}











