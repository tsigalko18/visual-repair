% !TEX root =  paper.tex

\section{Empirical Evaluation}\label{sec:evaluation}

We consider the following research questions:

\noindent
\textbf{RQ\textsubscript{1}:} How prevalent are the various breakage scenarios in practice?

\noindent
\textbf{RQ\textsubscript{2}:} How do visual-based and DOM-based test repair approaches compare in terms of effectiveness?

\noindent%\quad\hangindent=0.35cm
\textbf{RQ\textsubscript{3}:} this is research question 2

\noindent
In our study, the visual-based approach is represented by \tool, and the DOM-based approach is represented by \water.

\subsection{Subjects}\label{sec:subjects}

We selected XXX open source web applications for which Selenium test cases were available. Such applications span different application domains and have been used in the context of previous research on web testing~\cite{}.

\subsection{Procedure}\label{sec:procedure}

(1)~I can use existing test suites and try them on subsequent versions; \\
(2)~mutate existing test suites and inject faults; \\
(3)~user study \\
(4)~measure how effective the tool helps in the identification of the root cause, and how much time it allows to save

Although we do not present a detection algorithm, we our repair strategy indirectly contains a detection strategy based on visual trace execution differencing. We do evaluate it indirectly, while evaluating the number of correct/failed repairs.

\subsection{Results}\label{sec:results}