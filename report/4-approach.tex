%% !TEX root =  paper.tex
\section{Approach}\label{sec:approach}

\begin{itemize}
\item use SIFT to determine whether the image is present/absent
\item SIFT returns also an estimate of where the majority of keypoints has been found
\item Template matching usually returns multiple or inaccurate results
\item the idea is to filter the TM results with SIFT's
\item one try might be to use multiple images (i.e. both the perfectly cropped and the larger visual locator).
\item try to implement Chamfer Matching
\end{itemize}

\begin{itemize}
\item The scope of this paper is to repair test breakages. 
\item What about detection?
\item We focus on locators because they represent the main problem
\item The algorithm can detect and correct locator problems that pertains both to direct and propagated breakages.
\item We do not target silent breakages because we assume the software to be correct
\end{itemize}

The goal of our approach is to automatically find potential fixes that can repair a broken web test that was used to work correctly on a  version $k$ of the AUT, and that now fails when applied on a subsequent version $k+n$ (with $n>1$).

The focus of our technique is to repair \textit{locators}, that represent the main source of breakage.
Our technique can detect and correct locator problems that pertain to the breakage scenarios described in \autoref{sec:breakage-scenarios}. A main assumption of our work is that the two web applications object of our analysis are correct, i.e., do not contain bugs that make the tests fail. Thus, we ensure that when tests fail to execute is due to actual regressions of the tests. For this reason, in this work, we do not target silent breakages because we assume the software to be correct.

Formally, \ldots 

Our approach generates repairs using \ldots

We now introduce the steps of our approach

\head{Visual Trace Collection}

The first step consists

\head{Test Repair}


%\IncMargin{0.5em}
%\begin{algorithm}[h]
%\scriptsize
%\SetAlgoLined
%\KwResult{Write here the result}
% initialization\;
% \While{While condition}{
%  instructions\;
%  \eIf{condition}{
%   instructions1\;
%   instructions2\;
%   }{
%   instructions3\;
%  }
% }
% \caption{How to write algorithms}
%\end{algorithm}\DecMargin{1em}

\subsection{Implementation}\label{sec:implementation}

We implemented our approach in a tool called \tool, which is publicly available (URL omitted). 
The tool is written in Java, and supports Selenium test suites written in Java. However, our overall approach is more general and applicable to test suites developed using other programming languages or testing frameworks. 
\tool gets as input the path to the test suites, collects the visual execution traces by means of an AspectJ module, and runs the visual repair algorithms. 
\tool makes use of the traces to generate potential repairs and generates a list of repaired test cases.

\head{Requirements and Limitations}

A running requirement is that the browser needs to be displayed and the web elements rendered on the screen (i.e., headless browsers such as PhantomJS are currently not supported).

A limitation is that the tests numbers need to be synchronized between the two versions. \andrea{Rahul: how do we deal with that when we added/removed statements?}







