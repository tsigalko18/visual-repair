In the web domain, automated test suites are prone to break frequently as the web application under test evolves. Test engineers must spend precious time and resources to figure out (i)~the reason behind the breakage, and (ii)~a fix for it. In doing so, they arguably need to inspect the GUI of the web application, or replay a portion of the test in order to detect where the deviation from the normal behaviour occurred. 
Existing test repair technique are very limited because they do not support detection, nor they use any visual information of the application.
In this paper, we propose a novel web test breakage detection and repair solution, which logs a trace of relevant DOM and visual information from a test execution, which is subsequently used to perform online detection (and repair) of breakages, when it used for regression purposes. We implemented our solution in a tool, \tool, that leverages a combination of computer vision algorithms and crawling  techniques to support a large variety of breakage scenarios. Our empirical experiment on XXX test suites for YYY real-world web applications showed that \tool was able to repair TOT \% of the total number of breakages. The result is an effective runtime detection and self-repair technique that minimize the need for testers to surmise about where and how their tests break.