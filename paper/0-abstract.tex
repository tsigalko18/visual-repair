In the web domain, automated test suites are prone to frequent breakages as the web application under test evolves forcing test engineers to spend precious time to fix them. In doing so, they typically inspect the GUI of the web application, or replay a portion of the test in order to detect where the deviation from the normal behaviour occurred. 
Existing test repair techniques are very limited because they do not use any visual information of the application.
In this paper, we propose a novel web test breakage repair solution, which captures a trace of relevant DOM and visual information from a test execution, and uses it to perform runtime detection and repair of breakages in a regression testing scenario. We implemented our solution in a tool, \tool, that leverages a combination of computer vision algorithms and crawling  techniques to support a large variety of breakage scenarios. Our empirical experiment on 2672 test cases spanning 86 releases of four real-world web applications showed that \tool was able to repair up to 81\% of the total number of breakages, with an improvement of 40\% with respect to a state of the art technique. The result is an effective runtime detection and self-repair technique that minimizes the need for testers to surmise about where and how their tests break.